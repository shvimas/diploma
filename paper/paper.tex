\documentclass[oneside, final, 12pt]{article}
\usepackage[utf8]{inputenc}
\usepackage[russianb]{babel}
\usepackage{vmargin}
\usepackage{amsfonts}
\usepackage{amssymb}
\usepackage{amsmath}
\usepackage{xcolor}
\usepackage{hyperref}
\usepackage{indentfirst}
\usepackage{color}
\usepackage{listings} 
\usepackage{caption}
\usepackage{graphicx}
\setpapersize{A4}
\setmarginsrb{2cm}{1.5cm}{1cm}{1.5cm}{0pt}{0mm}{0pt}{13mm}

\DeclareCaptionFont{white}{\color{white}}

\DeclareCaptionFormat{listing}{\colorbox{gray}{\parbox{\textwidth}{#1#2#3}}}
% \captionsetup[lstlisting]{format=listing,labelfont=white,textfont=white}

\newenvironment{compactlist}{
    \begin{list}{{$\bullet$}}{
        \setlength\partopsep{0pt}
        \setlength\parskip{0pt}
        \setlength\parsep{0pt}
        \setlength\topsep{0pt}
        \setlength\itemsep{0pt}
    }
}{
    \end{list}
}

\begin{document}

\lstset{
language=Python,                  % язык для подсветки
basicstyle=\small\sffamily,       % размер и начертание шрифта для подсветки кода
numbers=left,                     % положение нумерации строк (слева/справа)
numberstyle=\tiny,                % размер шрифта для номеров строк
stepnumber=1,                     % размер шага между двумя номерами строк
numbersep=5pt,                    % как далеко отстоят номера строк от подсвечиваемого кода
backgroundcolor=\color{white},    % цвет фона подсветки (требуется package 'color')
showspaces=false,                 % показывать или нет пробелы специальными отступами
showstringspaces=false,           % показывать или нет пробелы в строках
showtabs=true,                    % показывать или нет табуляцию в строках
frame=single,                     % рисовать рамку вокруг кода
tabsize=4,                        % размер табуляции
captionpos=t,                     % позиция заголовка вверху [t] или внизу [b] 
breaklines=true,                  % автоматически переносить строки (да\нет)
breakatwhitespace=false,          % переносить строки только если есть пробел
escapeinside={\%*}{*)}            % если нужно добавить комментарии в коде
}


\thispagestyle{empty}
\centerline{\bf Московский Государственный Университет им. М.\,В.\,Ломоносова}
\vskip0.1cm
\centerline{Факультет вычислительной математики и кибернетики}
\centerline{Кафедра математической статистики}
\vskip0.1cm
\centerline{\hfill\hrulefill\hrulefill\hfill}

\begin{center}
  \includegraphics[width=0.5\linewidth]{img/msu_logo.jpg}
\end{center}

\vfill
\vfill
\large
\centerline{Новиков Дмитрий Александрович}
\vfill
\large
\begin{centering}
\vskip0.2cm
\Large
{\bf <<Сравнение трех моделей ценообразования опционов>>\\}
\end{centering}
\normalsize
\vfill
\vfill
\vfill
\centerline{ВЫПУСКНАЯ КВАЛИФИКАЦИОННАЯ РАБОТА}
\vfill
\begin{flushright}
Научный руководитель:\\
к.\,ф-м.\,н.\,Назаров\,Л.\,В.
\end{flushright}
\vfill
\vfill
\centerline{Москва 2018}

% Оглавление
\newpage
\tableofcontents

% Список основных обозначений
\newpage
\large
\leftline{\bf Список обозначений}
\leftline{\hrulefill}
\vskip0.25cm
\normalfont

Далее будут использованы следующие обозначения:\\
$\mathbb{E}\xi$~-- математическое ожидание с.в. $\xi$\\
$\mathbb{D}\xi$~-- дисперсия с.в. $\xi$\\
$\displaystyle \Phi(x)=\frac{1}{\sqrt{2\pi}}\int\limits_{-\infty}^{x}e^{-\frac{t^2}{2}}\,dt$~-- ф.р. стандартной нормальной с.в. \( \xi \sim \mathcal{N}(0, 1)\) \\
$W_t$~-- винеровский процесс\\
$S_t$~-- спот, рыночная цена базового актива на момент времени $t$. $S_0$ - текущая цена\\
$r$~-- безрисковая процентная ставка с непрерывным начислением\\
$q$~-- дивиденды с непрерывным начислением\\
$T$~-- время до погашения опциона\\
$K$~-- страйк, договорная цена базового актива\\
$C$~-- цена европейского колл-опциона\\
$P$~-- цена европейского пут-опциона\\
$c$~-- логарифм цены европейского колл-опциона

% Введение
\newpage
\section{Цель работы}
На данный момент существует множество разнообразных моделей ценообразования опционов, но они редко сравниваются между собой, кроме разве что модели Блэка-Шоулза, которая считается отправной точкой при описании почти любой новой модели оценки стоимости опционов. Ввиду очевидного несоответствия условий применимости модели Блэка-Шоулза к реальныи биржевым процессам (например, предполагаемое постоянство волатильности и процентной ставки) такие сравнения заведомо слабо информативны в отношении качества модели. \par

Цель настоящей работы состоит в проведении сравнительного анализа трех моделей определения цены опциона: Хестона, Variance Gamma и Finite Moment Log Stable. Сравниваются скорость калибрации параметров моделей, качество калибрации, качество прогнозирования, устойчивость откалиброванных параметров по времени. В кодовой реализации моделей используется подход, предложенный Карром (Peter Carr) и Маданом (Dilip B. Madan) в статье <<Option pricing using the fast Fourier transform>>\cite{FFT:paper}


\newpage
\section{Введение}
\subsection{Понятие опциона, цены опциона}
Сейчас опцион~-- производный финансовый инструмент, дающий право в будущем купить(колл-опцион) или продать(пут-опцион) базовый актив по фиксированной сегодня цене~-- является одним из основных инструментов на биржах ценных бумаг. Существует большое количетсво вариаций, но в данной работе исследоваться будут модели, оценивающие самый простой вид опционов: европейский колл-опцион (опцион на покупку, исполнить который можно в строго определенный момент времени). Пусть $S$~-- цена базового актива, $r$ и $q$~-- безрисковая процентная ставка и доходность от базового актива, $T$~-- время до погашения опциона, $K$~-- договорная цена. Тогда $C$ и $P$~-- цены колл и пут опционов, соответственно, связаны: 
\begin{equation}\label{put_call_parity}
C-P = S_T\,e^{-qT} - Ke^{-rT}
\end{equation}
Это соотношение, называемое Put-Call Parity, дает возможность рассматривать только колл-опционы, т.к. цена пут-опциона явно выводится из цены колл. \par
Перед тем как идти дальше, необходимо пояснить, что рассматриваются т.н. справедливые цены опционов, которые не дают ни одной стороне преимущества в том плане, что невозможен арбитраж~-- неотрицательный доход с положительной вероятностью при нулевой вероятности убытков. На бирже особенно применимы законы свободного рынка, что подразумевает выравнивание цен к одной справедливой, не дающей ни продавцу, ни покупателю такого явного денежного преимущества. Это и дает возможность определять стоимость опциона, как однозначную функцию от параметров $T$ и $K$ контракта и конечной цены $S_T$:
\begin{equation}\label{call_price}
C = \max(0, S_T - K)
\end{equation}
\begin{equation}\label{put_price}
P = \max(0, K - S_T)
\end{equation}
\begin{center}
  \includegraphics[width=1\linewidth]{img/call-put_price.png}
\end{center}

\subsection{Первые модели ценообразования опционов}
Первым исследовать такие справедливые цены опционов начал Башелье (Louis Jean-Baptiste Alphonse Bachelier), опубликовавший в 1900 году свою диссертацию <<The Theory of Speculation>>, в которой впервые применил продвинутую математику в теории финансов. В частности, он рассмотрел модель броуновского движения как описание цен базового актива, что позволило вывести искомую цену опциона.

Затем в 1973 Блэк (Fischer Black) и Шоулз (Myron Scholes) построили модель, в которой также движение цены базового актива обуславливалось винеровским процессом. Они показали, что при выполнении ряда предположений модели, спот-цена должна удовлетворять следующему стохастическому уравнению:
\[ 
dS_t = r S_t dt + \sigma S_t dW_t 
\]
Тогда можно вывести справедливую цену колл-опциона:
\[
C=S_0\Phi\left(\frac{\ln{\frac{S_0}{K}}+\frac{1}{2}\sigma^2T}{\sigma\sqrt{T}}\right)-Ke^{-rT}\Phi\left(\frac{\ln{\frac{S_0}{K}}-\frac{1}{2}\sigma^2T}{\sigma\sqrt{T}}\right)
\]
При этом важным параметром модели становится $\sigma$~-- стандартное отклонение винеровского процесса или волатильность. В модели Блэка-Шоулза волатильность является постоянной величиной. \par
Довольно скоро стало ясно, что модельные предположения (в частности, о постоянстве волатильности) оказались неверными для реальных рынков, и по их стопам пошли многие другие финансовые математики, дополнявшие, развивавшие и изменявшие модель Блэка-Шоулза. Так, в 1993 году Хестон (Steven Heston) предложил свою модель, в которой волатильность описывается уже своим винеровским процессом. В 1998 Мадан (Dilip B. Madan), Карр (Peter Carr) и Чанг(Eric C. Chang) предложили для описания движения спота использовать Variance Gamma процесс, обобщающий процесс броуновского движения, что позволило учесть изменчивость волатильности в случайные моменты времени. В 2003 Карр (Peter Carr) и Ву (Liuren Wu) заметили странное поведение т.н. <<улыбки волатильности>> (volatility smirk), которое противоречило большинству используемых моделей, и разработали Finite Moment Log Stable процесс, который объяснял наблюдаемые аномалии. Ниже будут даны краткие обзоры этих трех моделей.

\newpage
\subsection{Модель Хестона}
В 1993 году Хестон (Steven Heston) в своей работе <<A Closed-Form Solution for Options with Stochastic Volatility with Applications to Bond and Currency Options>> развил идеи Блэка и Шоулза, позволив волатильности \(\nu_t\) быть стохастической. В его модели спот-цена $S_t$ описывается стохастическим процессом:
\[dS_t=rS_tdt+\sqrt{\nu_t}S_tdW_t^S\]
где $\nu_t$~-- мгновенная дисперсия, задаваемая процессом CIR(Cox–Ingersoll–Ross):
\[d\nu_t=\kappa(\theta-\nu_t)dt+\sigma\sqrt{\nu_t}dW_t^\nu\]
При этом в модель добавляется параметр $\rho$, задающий корреляцию $W^S_t$ и $W^\nu_t$:
\[\mathbb{E}\left[dW_t^S\, dW_t^\nu\right]=\rho dt\]
\\
Параметры имеют следующий смысл:
\begin{itemize}
\item $r$~-- ожидаемая доходность базового актива
\item $\theta$~-- ожидаемое на бесконечности значение: 
  \( \mathbb{E} \left[\underset{t \rightarrow \infty}{\text{lim}} \nu_t \right] = \theta \)
\item $\kappa$~-- частота с которой $\nu_t$ возвращается к $\theta$ 
\item $\sigma$~-- волатильность волатильности, дисперсия $\nu_t$
\end{itemize}

Модель Хестона значительно ослабила условия Блэка-Шоулза, позволив волатильности меняться со временем, но, например, доходность осталась постоянной, за что эта модель также подвергалась критике. Существуют модификации, добавляющие зависимость от времени и стохастичность остальным параметрам модели, но стоит заметить, что даже пять параметров~-- уже довольно много, такое число степеней свободы довольно сильно сказывается на скорости и устойчивости параметров при калибрации, см. раздел 5.

\newpage
\subsection{Модель Variance Gamma}
Для того, чтобы определить процесс Variance Gamma (VG), определим \( b(t; \theta, \sigma) = \theta t + \sigma W_t \), как броуновское движение со сдвигом $\theta$ и волатильностью $\sigma$. Определим также гамма-процесс $\gamma(t; \mu, \nu)$ как процесс из независимых гамма-приращений на непересекающихся интервалах времени $(t, t + h)$. Тогда плотность такого приращения $g = \gamma(t+h; \mu, \nu) - \gamma(t; \mu, \nu)$ будет равна \[ f_h(g) = \left(\frac{\mu}{\nu}\right)^{\frac{\mu^2h} {\nu} } \frac{g^{\frac{\mu^2h}{\nu} - 1} \exp{\left(-\frac{\mu}{\nu}g\right)}}{\Gamma\left( \frac{\mu^2h}{\nu} \right)}, \,g > 0 \] Тогда определим VG процесс, как процесс броуновского движения в случайный момент времени, распределенный по гамма-закону: \[ X(t; \sigma, \nu, \theta) = b(\gamma(t, 1, \nu), \theta, \sigma) \] Этот процесс был представлен в 1998 году Маданом (Dilip B. Madan), Карром (Peter Carr) и Чангом (Eric C. Chang). Благодаря обобщению броуновского процесса VG процесс лучше подходит для описания цены актива на бирже ввиду частого изменения активности торгов, т.к. замена времени позволяет моделировать значение спота смесью нормальных законов. \par
Также процесс VG хорош тем, что выведены\cite[стр. 85]{VG:paper} явные формулы для первых четырех моментов распределения в момент времени $t$. \\
Пусть \( \mu^n = \mathbb{E}\left[ (X(t; \sigma, \nu, \theta) - \mathbb{E}[X(t; \sigma, \nu, \theta)])^n \right] \). Тогда:
\[ \mathbb{E}[X(t; \sigma, \nu, \theta)] = \theta t \]
\[ \mu^2 = (\theta^2 \nu + \sigma^2)t \]
\[ \mu^3 = (2 \theta^3 \nu^2 + 3 \sigma^2 \theta \nu)t \]
\[ \mu^4 = (3 \sigma^4 \nu + 12 \sigma^2 \theta^2 \nu^2 + 6 \theta^4 \nu^3)t + 
(3 \sigma^4 + 6 \sigma^2 \theta^2 \nu + 3 \theta^4 \nu^2)t^2 \]

 
\newpage
\subsection{Модель на основе Finite Moment Log Stable процесса}
В 2003 Карр (Peter Carr) и Ву (Liuren Wu) опубликовали работу <<The Finite Moment Log Stable Process and Option Pricing>>, в которой обозначили замеченную ими \lq\lq{}аномалию\rq\rq{} рыночных данных о ценах на опционы на индексный фьючерс S\&P 500~-- даже на больших временных горизонтах (более 2 лет) \lq\lq{}ухмылка\rq\rq{} предполагаемой волатильности (implied volatility smirk)\footnote{Улыбка (ухмылка) волатильности~-- график предполагаемой по модели Блэка-Шоулза волатильности out-of-the-money опционов колл и пут в зависимости от страйка при постоянном времени экспирации} не выравнивается, что было бы ожидаемо в предположениях о применимости центральной предельной теоремы (ЦПТ), которые использовались в большинстве моделей оценки стоимости опционов. Это привело авторов к тому, что процесс, описывающий реальные данные должен нарушать условия ЦПТ. При постороении своего процесса Карр и Ву использовали следующее стохастическое дифференциальное уравнение: 
\[ \frac{dS_t}{S_t} = (r - q)dt + \sigma dL_t^{\alpha, -1}, \alpha \in (1, 2)\] где \( L_t^{\alpha, \beta}~-\)стандартное $\alpha$-стабильное движение Леви\\(standardized
Levy $\alpha$-stable motion)\cite{Levy:paper}. Стоит отметить, что $L_t^{2, 0}$~-- стандартный винеровский процесс. 

\textbf{ ссылка на источник картинки?}

\begin{center}
  \includegraphics[width=1\linewidth]{img/implied_volatility.png}
\end{center}


\newpage
\section{Использование быстрого преобразования Фурье}
В 1999 Карр и Мадан показали, что в случае, когда характеристическая функция логарифма спота известна, для вычисления цены колл-опциона можно использовать быстрое преобразование Фурье. Не останавливаясь на деталях, приведем формулы для вычислений. Пусть $S_t$~-- спот на момент $t$, $K$~-- страйк, $T$~-- дата погашения опциона, $q_t(s)$~-- риск-нейтральная мера спота на момент $t$, $C_T(k)$~-- цена опциона со страйком $K = e^k$. Положим
\[ s_t = \ln{S_t} \]
\[ k = \ln{K} \]
\[ \phi_T(u) = \mathbb{E}\left[ \exp{ius_T} \right] \text{~-- хар. функция } s_T=\ln{S_T}\]
\[ C_T(k) = \int\limits_k^\infty e^{-rT}(e^s - e^k)q_T(s)ds \]
\[ c_T(k) = e^{\alpha k}C_T(k),\,\, \alpha > 0 \]
\[ \psi_T(v) = \int\limits_{-\infty}^{+\infty} e^{ivk} c_T(k) dk\]
Тогда можно показать, что 
\[ \psi_T(v) = \frac{e^{-rT}\phi_T(v - (\alpha + 1)i)}{\alpha^2 + \alpha - v^2 + i(2\alpha + 1)v} \]
\[ C_T(k) = \frac{e^{-\alpha k}}{\pi} \int\limits_0^\infty e^{-vk} \psi_T(v) dv \]

Правильный выбор параметра $\alpha$ очень важен для корректности вычислений. Карр и Мадан рекомендуют в качестве верхней границы допустимых значений $\alpha$ брать 
\( \frac{1}{4} \, \underset{\alpha > 0}{\mathrm{sup}}(\alpha \,\, | \,\, \mathbb{E}[S_T^{\alpha+1}] < \infty) \). \\
\textbf{дописать еще}


\newpage
\section{Постановка задачи калибрации}
В случае калибрации моделей ценообразования опционов процесс сводится к решению задачи оптимизации вида:
\[
f_\text{error}(M(p_1, ..., p_k, S, \bar K, T, r, q), P_\text{real}) \rightarrow \underset{p_1, ..., p_k}{\min}
\]
где $f_\text{error}(\cdot \, , \cdot)$~-- метрика ошибки приближения, $M$~-- модельная цена опционов,\\ $p_1$, ..., $p_k$~-- параметры модели, $S$~-- спот-цена, $\bar K$~-- набор страйков, $T$~-- время до погашения, $r$ и $q$~-- безрисковая процентная ставка и доходность, соответственно, $P_\text{real}$~-- рыночные цены опционов. В данной работе для оценки качества приближения использовалась метрика RMSE, т.е. \( f_\text{error}(\bar v_1, \bar v_2) = \sqrt{ \frac{1}{n} \sum\limits_{i = 1}^n{(\bar v_1^i - \bar v_2^i)^2} } \). 
\\
Для нахождения глобального минимума на области допустимых моделью параметров использовался алгоритм differential evolution\cite{DE:paper1}\cite{DE:paper2}. Стоит отдельно отметить, что использование быстрого преобразования Фурье\cite{FFT:paper} при моделировании цены опционов особенно хорошо сказывается при калибрации моделей с большим числом параметров, например, для модели Хестона это помогло сократить время оптимизации набора из 20 страйков более чем в 15 раз.

\newpage
\section{Калибрация моделей}
В этом разделе будут подробно рассмотрены полученные результаты по качеству калибрации параметров, скорости сходимости параметров к оптимальным, устойчивости откалиброваных параметров. 

\subsection{Качество калибрации}

\begin{center}
  \includegraphics[width=1\linewidth]{img/RMSE_quality.png}
\end{center}

Хорошо видно, что модель Хестона доминирует над VG и FMLS. При этом стоит отметить, что значение спота составляло около 1200 на протяжении дней измерения, что означает, что все модели достаточно хорошо подгоняются под реальные данные (средняя погрешность порядка 0.1\%).

\newpage
\subsection{Скорость калибрации}
Очевидно, что нет никакого смысла приводить абсолютные временные замеры. В качестве единицы отсчета возьмем среднее время калибрации модели Блэка-Шоулза на машине автора. Первая таблица показывает среднее время расчета цены K опционов с использованием FFT\cite{FFT:paper}. Вторая таблица приводит среднее время калибрации на реальных данных на K страйках, используя алгоритм differential evolution\cite{DE:paper1}.


\begin{table}[h!]
  \begin{center}
    \caption{Средняя скорость расчета цены}
    \label{tab:table1}
    \begin{tabular}{c|c|c|c}
      \textbf{K} & \textbf{Heston} & \textbf{VG} & \textbf{FMLS}\\
      \hline
      1   & 4.4 & 2.78 & 2.67\\
      10  & 9.62 & 8.08 & 7.96\\
      25  & 12.5 & 11.37 & 11.44\\
      50  & 24.58 & 23.47 & 22.97\\
      100 & 39.63 & 40.52 & 37.02\\
      200 & 61.39 & 60.85 & 58.49\\
    \end{tabular}
  \end{center}
\end{table}

\begin{table}[h!]
  \begin{center}
    \caption{Средняя скорость калибрации}
    \label{tab:table2}
    \begin{tabular}{c|c|c|c}
      \textbf{K} & \textbf{Heston} & \textbf{VG} & \textbf{FMLS}\\
      \hline
      1   & 23 & 11 & 7 \\
      10  & 155 & 89 & 21\\
      25  & 245 & 128 & 29\\
      50  & 470 & 284 & 57\\
      100 & 865 & 488 & 102\\
      200 & 1126 & 730 & 156\\
    \end{tabular}
  \end{center}
\end{table}


\subsection{Устойчивость оптимальных параметров}
При прогнозе цен опционов на уже откалиброванных параметрах, важно уделить внимание устойчивости~-- средним колебаниям ($ \sqrt{\mathbb{D}\xi}\;/\;\mathbb{E}|\xi| $) параметров при повторных калибрациях. Оказалось, что модели VG и FMLS крайне устойчивы: вариация пренебрежимо мала по сравнению со значением параметра.

\begin{center}
  \includegraphics[width=1\linewidth]{img/vg_stable.png}
\end{center}

\begin{center}
  \includegraphics[width=1\linewidth]{img/ls_stable.png}
\end{center}


\newpage
Анализ же модели Хестона показывает, что устройчив лишь параметр $\rho$~-- корреляция винеровских процессов.
Безусловно, это необходимо учитывать при построении прогнозов с помощью откалиброванной модели. 

\begin{center}
  \includegraphics[width=1\linewidth]{img/heston_stable.png}
\end{center}

Отметим, что модель Блэка-Шоулза также очень устойчива (значение вариации показано без фактора $10^{-7}$)

\begin{center}
  \includegraphics[width=1\linewidth]{img/bs_stable.png}
\end{center}

\newpage
\section{Анализ взаимозаменяемости моделей}
Для того, чтобы оценить степень взаимозаменяемости моделей, было проведено следующее:
\begin{enumerate}
\item Область параметров полученных при калибрации VG была окружена кубом, внутри которого построена сетка из 400 точек. Таким образом были получены параметры, похожие на "настоящие". Модель VG была выбрана в качестве стартовой, т.к. ее параметры имеют понятную интерпретацию, см. (\textbf{вставить ссылку на VG})
\item По этим параметрам были построены модельные цены, опять же, близкие к реальным, на которых были откалиброваны модели Хестона и FMLS, на каждой из которых затем были откалиброваны две другие. Таким образом были получены 6 калибраций, по 2 на каждую из моделей. Описанный процесс является симметричным в том смысле, что каждая из моделей была и калибрующей, и калибруемой для двух других.
\item После описанной процедуры для каждой из 400 точек, задающих исходные параметры VG, имеем 6 метрик качества \( S_{m1, m2} \), где \( \text{m1} \not = \text{m2} \). Поставим в соответствие каждой модели m число \( S_m = \frac{\sum\limits_{m1 = m, m2 \not = m}{S_{m1, m2}}}{\sum{S_{m1, m2}}} \). Тогда модель с наименьшим значением $S_m$ является наилучшей.
\end{enumerate}

\textbf{добавить картинки}


\newpage
\section{Сравнение по AIC}
Сравним также модели по информационному критерию Акаике (AIC)\cite{AIC:paper}: \( \text{AIC} = 2k - 2\ln{L} \), где $k$~-- число параметров модели,  $L$~-- максимизированное значение функции правдоподобия. Отметим, что абсолютное значение AIC не имеет смысла, и наилучшей считается модель с наименьшим значением AIC. Можно показать, что в предположениях об одинаковой длине выборок $n$, нормальной и независимой распределенности ошибок модели $\varepsilon_i$ \( AIC = 2k +  n\ln{\sum\limits_{i = 1}^n}{\varepsilon_i^2}\)
\begin{center}
  \includegraphics[width=1\linewidth]{img/AIC_187.png}
\end{center}


\newpage
\section{Выводы}


\newpage
\section{Заключение}
Подводя итог проделанной работе, хочется отметить несколько моментов:
\begin{itemize}
\item Ввиду технических ограничений все калибрации проводились по 187 рабочим дням 2011 года. Уже после выполнения работы были получены данные за 91 рабочий день 2017/2018 года, при прогоне аналогичных тестов и метрик все выводы остались прежними. Безусловно, хотелось бы увеличить охват исследуемых дней~-- это остается для будущих работ.
\item В последующих работах стоит обратить внимание на более современные модели оценивания опционов, как, например, в работе Grzelak, L.A. \& Oosterlee, C.W. (2011), \lq\lq{}On the Heston Model with Stochastic Interest Rates\rq\rq{}, а так же на методы, основанные на применении нейронных сетей.
\item Отдельным интересным направлением для изучения является выбор алгоритма глобальной оптимизации для калибрации параметров моделей. Выбор автора, алгоритм differential evolution\cite{DE:paper1}, был основан в основном лишь на общей применимости алгоритма и на удовлетворительных результатах калибрации с ним.
\end{itemize}



\newpage
\begin{thebibliography}{00}
\bibitem{Heston:paper} Steven L. Heston (1993)
\emph{A Closed-Form Solution for Options with Stochastic Volatility with Applications to Bond and Currency Options}

\bibitem{LS:paper} Peter Carr, Liuren Wu (2003)
\emph{The Finite Moment Log Stable Process and Option Pricing}

\bibitem{VG:paper} Dilip B. Madan, Peter P. Carr, Eric C. Chang (1998)
\emph{The Variance Gamma Process and Option Pricing}

\bibitem{FFT:paper} Peter Carr, Dilip B. Madan (1999)
\emph{Option valuation using the fast Fourier transform}

\bibitem{DE:paper1} Rainer Storn, Kenneth Price (1997)
\emph{Differential Evolution — A Simple and Efficient Heuristic for Global Optimization over Continuous Spaces.}

\bibitem{DE:paper2} K. Price, R. Storn, J. Lampinen (2005)
\emph{Differential Evolution: A Practical Approach to Global Optimization}

\bibitem{AIC:paper} Akaike, H. (1974)
\emph{A new look at the statistical model identification}

\bibitem{Levy:paper} Gennady Samorodnitsky, Murad S. Taqqu (1994)
\emph{Levy Measures of Infinitely Divisible Random Vectors and Slepian Inequalities}

\end{thebibliography}

\end{document}